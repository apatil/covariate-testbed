\documentclass[a4paper]{article}
\usepackage{fullpage}
\usepackage{epsfig}
\usepackage{pdfsync} 
\usepackage{amsfonts}
\usepackage{amsmath} 
\begin{document}

\title{Blah}
\author{anand}
\maketitle

\section{Single covariate}

The marginal precision matrix of data whose mean has prior variance $uwu^T$, with $w$ being the prior variance of the coefficient of the covariate and and $u$ a column vector containing the covariate's value at the datapoints, is
\begin{eqnarray*}
(z+uwu^T)^{-1}=z^{-1}- \frac{z^{-1}uu^Tz^{-1}}{1/w+u^Tz^{-1}u}.
\end{eqnarray*}
Here $z$ is the covariance matrix of the data given the coefficients. When $w$ is absolutely huge, $1/w$ is about zero and you get 
\begin{eqnarray*}
\lim_{w\rightarrow\infty}(z+uwu^T)^{-1}=z^{-1}-\frac{z^{-1}uu^Tz^{-1}}{u^Tz^{-1}u}.
\end{eqnarray*}

\section{Many covariates}
\label{sec:many_covariates}

Now say there are $n$ covariates, so the marginal covariance is
\begin{eqnarray*}
    z+\sum_k u_k w u_k^T.
\end{eqnarray*}
Express $\sum_k u_k u_k^T$ as $v dw v^T$, the eigenvector-eigenvalue decomposition. Applying the matrix inversion lemma gives
\begin{eqnarray*}
    z^{-1}-z^{-1}v(d^{-1}/w +v^Tz^{-1}v)^{-1}v^Tz^{-1},
\end{eqnarray*}
and taking the limit as above gives
\begin{eqnarray*}
    z^{-1}-z^{-1}v(v^Tz^{-1}v)^{-1}v^Tz^{-1}.
\end{eqnarray*}
If $v$ is full-rank, this is unfortunately zero. In geostats it's never going to be full-rank... but what if it were? Does that mean something is nonidentifiable? Does the posterior of $z$ exist at all?

\end{document}
